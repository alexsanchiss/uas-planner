\documentclass[12pt,a4paper]{article}
\usepackage[utf8]{inputenc}
\usepackage{geometry}
\geometry{margin=2.5cm}
\usepackage{hyperref}
\usepackage{longtable}
\usepackage{array}
\usepackage{listings}
\usepackage{xcolor}

% JSON listing style
\lstdefinestyle{json}{
    basicstyle=\ttfamily\scriptsize,
    breaklines=true,
    frame=single,
    backgroundcolor=\color{gray!10},
    showstringspaces=false
}

% Prints paths/URLs in teletype and allows line breaks at separators.
% Used inside tables to avoid overflow when showing endpoints.
\newcommand{\icdendpoint}[1]{\begingroup\small\path{#1}\endgroup}

\title{ SPIRIT \\ \small{ICD - V\_{2.2.0}}}
\author{U-Plan Preparation Service (UPPS) \\ Álex Sanchis}
\date{\today}

\begin{document}
\maketitle

\section*{Resumen}
Este documento describe la(s) interfaz(ces) del sistema \textbf{UPPS} (parte de \textbf{U-Plan Preparation Service, UPPS}) dentro del contexto del proyecto \textbf{SPIRIT}. En particular, especifica:
\begin{itemize}
    \item La interfaz entre el \textbf{cliente web} (operador UAS) y el \textbf{backend UPPS} (API HTTP de Next.js).
    \item La integración del backend con servicios externos: \textbf{FAS} (\textit{Flight Authorization Service}) y \textbf{Geoawareness Service}.
    \item La integración con el \textbf{Geoawareness Service} mediante un \textbf{proxy SSE} en el backend para obtención de geozonas y datos de U-space en tiempo real (sin conexión directa desde el navegador).
    \item La coordinación del procesamiento distribuido de trayectorias mediante un \textbf{daemon de asignación} (\texttt{traj-assigner.js}) y \textbf{máquinas de proceso} (workers) que ejecutan el proyecto externo \texttt{traj-runner}.
    \item El sistema de \textbf{regeneración automática de volúmenes de operación} para planes procesados.
    \item El \textbf{workflow guiado de 5 pasos} para preparación, procesamiento y autorización de planes de vuelo en la página \textbf{Plan Authorization}.
    \item Generación interactiva de planes de vuelo en la página \textbf{Plan Definition}.
    \item Sistema de \textbf{visualización interactiva} con mapas Leaflet para trayectorias, waypoints y geozonas.
\end{itemize}

El objetivo de la interfaz es permitir que un usuario cargue planes de vuelo (típicamente exportados desde QGroundControl o generados directamente en la aplicación), que dichos planes se procesen para generar trayectorias (CSV) y, finalmente, que esas trayectorias puedan convertirse a \textbf{U-Plan} para su evaluación de geoawareness y envío al servicio de autorización (FAS).

\subsection*{Novedades en V2.0.0 SPIRIT}
\begin{itemize}
    \item \textbf{Mejora general UI/UX}: Nuevo esquema visual, mejora de experiencia de usuario, optimización para escritorio y móvil.
    \item \textbf{Sistema de temas dinámico}: Soporte completo para temas claro/oscuro con cambio instantáneo y persistencia en localStorage.
    \item \textbf{Normalización de tiempos}: Las trayectorias CSV se normalizan automáticamente para iniciar en t=0s, independientemente del tiempo inicial de simulación.
    \item \textbf{Regeneración automática de volúmenes}: Sistema que verifica cada 30 segundos planes procesados sin volúmenes y los regenera automáticamente.
    \item \textbf{Optimización del flujo de volúmenes}: Eliminada la regeneración redundante de volúmenes antes del envío a FAS cuando ya existen, reduciendo notificaciones duplicadas y carga de procesamiento.
    \item \textbf{Viewer de trayectorias}: Visualización interactiva de trayectorias con mapas Leaflet, sustituyendo la descarga directa de CSV.
    \item \textbf{Visualización mejorada de volúmenes de operación}:
    \begin{itemize}
        \item Todos los volúmenes se muestran simultáneamente en el mapa (no solo los activos).
        \item Volúmenes activos destacados en morado brillante (40\% opacidad).
        \item Volúmenes inactivos mostrados en gris semi-transparente (15\% opacidad) para contexto.
        \item Control de tiempo permite navegar entre ventanas temporales.
        \item Carga fresca de datos desde API antes de cada visualización para garantizar sincronización.
    \end{itemize}
    \item \textbf{Geoawareness mejorado}: Auto-centrado de mapas en trayectorias, visualización con marcadores reducidos y overlays optimizados.
    \item \textbf{Gestión avanzada de folders}: Drag-and-drop para organización de planes, sección de planes huérfanos.
    \item \textbf{Timestamps dinámicos en U-Plan}: Los campos \texttt{creationTime} y \texttt{updateTime} se actualizan automáticamente al momento del envío a FAS.
    \item \textbf{Workflow optimizado}: Interfaz guiada de 5 pasos con indicadores de progreso, bloqueo de edición post-procesamiento y validaciones mejoradas.
    \item \textbf{Arquitectura modular mejorada}: Componentes reutilizables organizados por funcionalidad con hooks especializados.
\end{itemize}

\subsection*{Novedades en V2.2.0}

\subsubsection*{Sprint 1 (post v2.1.0) --- Robustez, UX y mejoras de flujo}
\begin{itemize}
    \item \textbf{Eliminación de validación DNS en registro} (Feature 1): Se elimina la función \texttt{validateEmailDomain()} del endpoint de registro, que bloqueaba dominios corporativos con configuraciones DNS no estándar. Solo se mantiene la validación de formato Zod.
    \item \textbf{Caducidad del código de verificación reducida a 3 minutos} (Feature 2): Código de verificación de email válido 3 minutos (antes 24~h); doble-verificación safeguard (\texttt{emailVerified: false}); mensaje de cooldown corregido en \texttt{/api/auth/resend-verification}.
    \item \textbf{Fix tema del sistema y toasts} (Feature 7): Tema por defecto basado en preferencia del SO; toasts totalmente opacos en ambos temas.
    \item \textbf{Validación de archivo U-Plan en carga} (Feature 8): Toast de error detallado si el archivo tiene extensión incorrecta, no es JSON válido, o carece de \texttt{operationVolumes}.
    \item \textbf{Campos de perfil de usuario} (Feature 9): Nuevos campos en BD \texttt{firstName}, \texttt{lastName}, \texttt{phone}; nuevos endpoints \texttt{GET/PATCH /api/user/profile}; \texttt{UplanFormModal} se pre-rellena automáticamente con los datos del perfil.
    \item \textbf{Notificación por email al cancelar plan aprobado} (Feature 10): Se envía email al usuario cuando se elimina un plan con \texttt{authorizationStatus="aprobado"}.
    \item \textbf{Botón OK y tecla Enter en edición de waypoints} (Feature 4): Los popups de edición de lat/lon incluyen botón OK y soporte de Enter para confirmar sin cerrar el teclado.
    \item \textbf{Botón ``Añadir Waypoint'' en la barra lateral} (Feature 5): Formulario inline para introducir coordenadas manualmente sin hacer click en el mapa.
    \item \textbf{Prevención de zoom reset en transiciones SCAN} (Feature 3): Se elimina \texttt{scanHandlerVersion} de la key de \texttt{PlanMap}, evitando el remontaje del mapa Leaflet en cada paso del flujo SCAN.
    \item \textbf{Generación automática de volúmenes antes de descargar U-Plan} (Feature 6): Si el U-Plan no tiene volúmenes de operación, se generan automáticamente antes de lanzar la descarga.
    \item \textbf{Visualización de denegación mejorada --- coloreado selectivo} (Feature 12): Solo los volúmenes geométricamente conflictivos se colorean en rojo (intersección 2D de polígonos); se muestra la razón FAS parseada con número de volúmenes y geozonas conflictivos.
    \item \textbf{Restricciones regulatorias EU para clase UAS} (Feature 13): Tabla de restricciones C0--C6 (MTOM máximo y dimensión máxima) validada en \texttt{PlanGenerator} y \texttt{UplanFormModal}.
    \item \textbf{Visor 3D de volúmenes de operación U-Plan con CesiumJS} (Feature 11): Nuevo componente \texttt{Cesium3DModal} con globe OSM Buildings; botón ``View 3D'' en \texttt{FlightPlansUploader}; nuevo endpoint \texttt{GET /api/cesium/token}.
\end{itemize}

\subsubsection*{Sprint 2 --- Nuevas Funcionalidades (visualización 3D, trayectorias, contacto)}
\begin{itemize}
    \item \textbf{Edición de perfil en /profile} (Feature 14): La edición de \texttt{firstName}, \texttt{lastName} y \texttt{phone} se traslada de \texttt{/settings} a \texttt{/profile}; la página de ajustes conserva solo tema, idioma y notificaciones.
    \item \textbf{Corrección de altitud AGL en trayectorias} (Feature 15): Al convertir trayectoria CSV a U-Plan se resta \texttt{plannedHomePosition[2]} de todos los valores \texttt{Alt}; los visores 3D usan \texttt{RELATIVE\_TO\_GROUND}.
    \item \textbf{Slider 4D en el visor 3D U-Plan} (Feature 16): Barra de tiempo con play/pause en \texttt{Cesium3DModal}; volúmenes activos resaltados, inactivos atenuados; velocidades 1×, 2×, 5×, 10×.
    \item \textbf{Modal unificado de resultado de autorización FAS 2D/3D} (Feature 18): Nuevo componente \texttt{AuthorizationResultModal} con pestañas 2D/3D; muestra planes aprobados y denegados; sección ``Parsed FAS Reason'' en lenguaje natural.
    \item \textbf{Mapa 3D de denegación con volúmenes de geozonas} (Feature 17): Nuevo componente \texttt{Denial3DModal}; volúmenes operativos en gris/rojo más prismas 3D semitransparentes de geozonas conflictivas con sus límites verticales.
    \item \textbf{Página de Contacto con sistema de tickets} (Feature 19): Formulario con Asunto, Categoría y Descripción; nuevo endpoint \texttt{POST /api/contact} que genera número de ticket \texttt{UPPS-AAAAMMDD-XXXX}; email de confirmación al usuario y notificación al equipo de soporte.
    \item \textbf{Visor 3D de trayectoria con dron animado} (Feature 20): Nuevo componente \texttt{Trajectory3DViewer} (Cesium); trayectoria como polilínea 3D semitransparente; slider temporal + play/pause; modelo de dron interpolado sobre la trayectoria.
\end{itemize}

\subsubsection*{Sprint 3 --- Polish, correcciones y arquitectura}
\begin{itemize}
    \item \textbf{Renombrado de páginas principales}:
    \begin{itemize}
        \item \texttt{/plan-generator} renombrado a \texttt{/plan-definition} (\textit{Plan Definition}).
        \item \texttt{/trajectory-generator} renombrado a \texttt{/plan-authorization} (\textit{Plan Authorization}).
        \item Los componentes internos \texttt{plan-generator/} han pasado a \texttt{plan-definition/}.
    \end{itemize}
    \item \textbf{Navegación directa Plan Definition $\rightarrow$ Plan Authorization} (Sprint 3 \#21):
    \begin{itemize}
        \item Tras subir un plan desde Plan Definition, la app navega automáticamente a \texttt{/plan-authorization?selectPlan=<id>}.
        \item Plan Authorization lee el parámetro \texttt{selectPlan} y auto-selecciona el plan recién importado.
    \end{itemize}
    \item \textbf{Sincronización del campo \texttt{state} del U-Plan con FAS} (Sprint 3 \#22):
    \begin{itemize}
        \item El handler \texttt{PUT /api/fas/<externalResponseNumber>} actualiza ahora el campo \texttt{state} dentro del JSON del U-Plan almacenado en BD, reflejando el estado comunicado por FAS (\texttt{ACCEPTED}, \texttt{WITHDRAWN}, etc.).
    \end{itemize}
    \item \textbf{Corrección de tema oscuro en InfoBox de Cesium} (Sprint 3 \#23):
    \begin{itemize}
        \item Se inyecta CSS en el iframe de \texttt{viewer.infoBox.frame} para forzar fondo oscuro y texto claro, eliminando el problema de texto blanco sobre fondo blanco en tema oscuro.
        \item Aplicado en: \texttt{Cesium3DModal}, \texttt{AuthorizationResultModal}, \texttt{Denial3DModal}, \texttt{Trajectory3DViewer}.
    \end{itemize}
    \item \textbf{Corrección del visor 3D de trayectoria} (Sprint 3 \#24):
    \begin{itemize}
        \item El dron ahora se actualiza correctamente durante la reproducción (usando ref en lugar de estado React para la lógica del RAF tick).
        \item Unificación de referencia de altura: trayectoria, waypoints y dron usan \texttt{RELATIVE\_TO\_GROUND} (AGL) de forma consistente, eliminando la doble ruta visible.
    \end{itemize}
    \item \textbf{Proxy SSE para Geoawareness en el backend} (Sprint 3 \#25 + mejora posterior):
    \begin{itemize}
        \item Arquitectura cambiada: el \textbf{navegador ya no abre WebSocket directamente} al servicio Geoawareness.
        \item El backend (Next.js) abre la conexión \texttt{ws://} internamente y retransmite cada mensaje al navegador como un evento SSE (\textit{Server-Sent Events}) via \texttt{GET /api/geoawareness/stream/\{uspaceId\}}.
        \item Esto elimina los errores de \textit{mixed-content} en despliegues HTTPS sin requerir TLS en el servicio Geoawareness.
        \item Solo se requiere la variable de entorno \textbf{servidor} \texttt{GEOAWARENESS\_SERVICE\_IP}; se depreca \texttt{NEXT\_PUBLIC\_GEOAWARENESS\_SERVICE\_IP}.
    \end{itemize}
    \item \textbf{Tooltips de ayuda en el formulario U-Plan} (Sprint 3 \#26):
    \begin{itemize}
        \item Nuevo componente reutilizable \texttt{FieldHelp} (icono \texttt{?}) con tooltip al hacer click.
        \item Explica acrónimos de aviación: SAC, SIC, MTOM, VLOS, BVLOS, SAIL, etc.
        \item Añadido en \texttt{UplanFormModal} y \texttt{PlanGenerator}/\texttt{PlanDefinition}.
    \end{itemize}
    \item \textbf{Corrección de activos Cesium en servidor} (fix buildtime):
    \begin{itemize}
        \item El script \texttt{postinstall} y el nuevo hook \texttt{prebuild} copian \textbf{todo} el directorio \texttt{Build/Cesium/} (incluyendo \texttt{Cesium.js}) a \texttt{public/cesium/}, corrigiendo el error 404 en despliegues donde \texttt{public/cesium} está en \texttt{.gitignore}.
    \end{itemize}
\end{itemize}

\subsection*{Novedades en V2.1.0}
\begin{itemize}
    \item \textbf{Sistema de gestión de emails con MailerSend}:
    \begin{itemize}
        \item Verificación de email con código de 6 dígitos y link de activación (24h validez).
        \item Reset de contraseña con token seguro (1h validez).
        \item Notificaciones automáticas de resultado de autorización FAS con UPLAN adjunto.
        \item Configuración mediante variables de entorno: \texttt{MAILERSEND\_API\_KEY}, \texttt{MAILERSEND\_SENDER\_EMAIL}.
    \end{itemize}
    \item \textbf{Importación de UPLANs externos}:
    \begin{itemize}
        \item Drag-and-drop de archivos .json con formato UPLAN directamente en carpetas.
        \item Botón de upload visible con icono de subida (verde) en cabecera de carpetas.
        \item Validación automática de estructura UPLAN (operationVolumes requeridos).
        \item Sin regeneración de volúmenes: UPLANs externos se envían directamente al FAS.
        \item Detección automática basada en ausencia de \texttt{csvResult}.
    \end{itemize}
    \item \textbf{Selector de U-Space para UPLANs externos}:
    \begin{itemize}
        \item Modal interactivo con mapa Leaflet mostrando todos los U-Spaces disponibles.
        \item Click en mapa selecciona el U-Space más pequeño que contiene el punto.
        \item Visualización de polígonos con colores: azul (disponible), naranja (seleccionado).
        \item Tooltip con nombre del U-Space al pasar el ratón.
        \item Asignación automática de \texttt{uspace\_identifier} en \texttt{geoawarenessData}.
    \end{itemize}
    \item \textbf{Visualización de denegaciones en mapa}:
    \begin{itemize}
        \item Modal específico para mostrar por qué un plan fue denegado.
        \item Volúmenes conflictivos en rojo, volúmenes válidos en verde.
        \item Geozonas conflictivas resaltadas en el mapa.
        \item Parsing robusto de múltiples formatos de respuesta FAS:
        \begin{itemize}
            \item Formato A: objetos con \texttt{\{ordinal, status\}}.
            \item Formato B: arrays de índices numéricos directos.
            \item Formato C: sin campo volumes, todos los volúmenes marcados como conflictivos.
        \end{itemize}
        \item Botón "View Denial on Map" visible cuando \texttt{authorizationStatus="denegado"}.
    \end{itemize}
    \item \textbf{Edición de waypoints en SCAN patterns}:
    \begin{itemize}
        \item Popups editables en puntos de takeoff, landing y vértices del polígono.
        \item Campos de input para latitud y longitud con actualización en tiempo real.
        \item Estilos compatibles con tema claro/oscuro usando CSS variables.
        \item Actualización bidireccional: drag-and-drop o edición manual de coordenadas.
    \end{itemize}
    \item \textbf{Mejoras en tema oscuro}:
    \begin{itemize}
        \item Popups de Leaflet con estilos globales para tema oscuro/claro.
        \item ScanPatternGeneratorV2 completamente adaptado con CSS variables (40+ reemplazos).
        \item Eliminación de clases hardcoded \texttt{bg-zinc-*}, \texttt{text-zinc-*}, \texttt{border-zinc-*}.
        \item Uso consistente de \texttt{var(--bg-primary)}, \texttt{var(--text-primary)}, etc.
    \end{itemize}
    \item \textbf{Botón de descarga de U-Plan}:
    \begin{itemize}
        \item Botón "Download U-Plan" visible después de procesamiento si existe \texttt{uplan}.
        \item Descarga directa del JSON con nombre \texttt{uplan\_<planname>.json}.
        \item Ubicado junto a los botones de acción (Process, Geoawareness, Authorize).
    \end{itemize}
    \item \textbf{Manejo robusto de errores 500 del FAS}:
    \begin{itemize}
        \item Sistema de reintentos automático: hasta 5 intentos con backoff exponencial (1s, 2s, 4s, 8s, 16s).
        \item Diferenciación de errores: 500 (reintento), 400/404 (marcar como denegado).
        \item Si todos los reintentos fallan con 500: plan vuelve a estado \texttt{"sin autorización"}.
        \item Respuesta HTTP 503 (Service Unavailable) al cliente para distinguir de errores reales.
        \item Toast informativo: "FAS temporarily unavailable. Please try again later."
    \end{itemize}
    \item \textbf{Correcciones de formato para UPLANs externos}:
    \begin{itemize}
        \item \texttt{scheduledAt} convertido a ISO-8601 con timezone (añade 'Z' si falta).
        \item Validación de \texttt{timeBegin} en operationVolumes antes de inserción en BD.
        \item Prevención de errores de Prisma DateTime al importar UPLANs externos.
    \end{itemize}
    \item \textbf{Mejora en flujo de selección de U-Space}:
    \begin{itemize}
        \item Apertura inmediata del viewer de geoawareness tras selección (sin segundo click).
        \item Llamada directa a API de geoawareness sin depender de actualización de estado React.
        \item Reducción de timing issues en flujo de usuario.
    \end{itemize}
\end{itemize}

\section{Alcance de la Interfaz}
\begin{itemize}
    \item \textbf{Cliente web (navegador)}: interfaz React/Next.js utilizada por el operador para cargar planes, organizar carpetas, lanzar procesamiento, descargar CSV y gestionar el envío a autorización.
    \item \textbf{Backend UPPS (Next.js API)}: endpoints HTTP bajo \icdendpoint{/api/*} que realizan operaciones CRUD sobre planes, carpetas y máquinas, y exponen endpoints especializados para \textbf{U-Plan} y \textbf{geoawareness}.
    \item \textbf{Base de datos MySQL}: persistencia de usuarios, carpetas, planes, resultados CSV y estado de máquinas (gestionada con Prisma).
    \item \textbf{Daemon de asignación de trabajos}: \texttt{traj-assigner.js} (Node.js) asigna planes en cola a máquinas disponibles mediante actualización de registros en BD.
    \item \textbf{Workers de procesamiento}: máquinas/VMs externas (no incluidas en este repositorio) que ejecutan \texttt{traj-runner} y escriben resultados en BD (tabla \texttt{csvResult}) y estado del plan.
    \item \textbf{Servicios externos}:
    \begin{itemize}
        \item \textbf{FAS} (Flight Authorization Service): recibe el U-Plan generado y posteriormente notifica el resultado al UPPS.
        \item \textbf{Geoawareness Service}: proporciona datos de U-spaces y geozonas. El backend UPPS abre una conexión WebSocket interna a \texttt{ws://<GEOAWARENESS\_SERVICE\_IP>/ws/gas/\{USPACE\_ID\}} y retransmite los mensajes al cliente web como eventos SSE a través de \texttt{GET /api/geoawareness/stream/\{uspaceId\}}. El navegador nunca se conecta directamente al servicio Geoawareness.
    \end{itemize}
    \item (Opcional) \textbf{noVNC/QGroundControl}: el frontend incluye un componente que embebe un \texttt{iframe} apuntando a \texttt{http://localhost:6080/} para visualizar/controlar una instancia VNC de QGroundControl.
\end{itemize}

\noindent\textbf{Fuera de alcance:} telemetría en tiempo real, control de aeronaves, enlaces MAVLink, o envío formal a autoridades (objetivos futuros descritos en \texttt{README.md}, no implementados completamente en el estado actual del repositorio).

\section{Entradas y Salidas}
\subsection*{Entradas}
Las entradas al sistema UPPS se reciben principalmente como solicitudes HTTP (JSON) al backend, así como respuestas/callbacks de servicios externos y actualizaciones en base de datos realizadas por procesos internos.

\subsubsection*{Entradas desde sistemas externos (FAS y Geoawareness)}
\begin{longtable}{|p{5cm}|p{2.5cm}|p{2cm}|p{6cm}|}
\hline
	\textbf{Campo} & \textbf{Tipo de dato} & \textbf{Formato} & \textbf{Descripción} \\
\hline
\endfirsthead
\hline
	\textbf{Campo} & \textbf{Tipo de dato} & \textbf{Formato} & \textbf{Descripción} \\
\hline
\endhead
\hline
\endfoot
\hline
\endlastfoot
FAS callback: endpoint & String & Ruta HTTP & Callback de FAS hacia UPPS: \texttt{PUT }\icdendpoint{/api/fas/<externalResponseNumber>}. \\
\hline
FAS callback: state & String & Texto & Campo \texttt{state} recibido en el callback. Si \texttt{state=ACCEPTED} se marca \texttt{authorizationStatus="aprobado"}; en caso contrario \texttt{"denegado"}. \\
\hline
FAS callback: message & String/Null & Texto & Campo \texttt{message} recibido en el callback; se persiste como \texttt{authorizationMessage}. \\
\hline
Geoawareness SSE: endpoint & String & URL HTTP & El cliente web se conecta al proxy del backend: \texttt{GET /api/geoawareness/stream/\{USPACE\_ID\}}. El backend abre internamente \texttt{ws://<IP>/ws/gas/\{USPACE\_ID\}} y retransmite mensajes como SSE. \\
\hline
Geoawareness SSE: message & Object & JSON & Evento SSE (tipo \texttt{data}) con el mensaje JSON del servicio Geoawareness. Estructura de 3 bloques: \texttt{uspace\_identifier}, \texttt{uspace\_data}, y \texttt{geozones} (FeatureCollection). Ver sección \ref{sec:geoawareness-ws-format}. \\
\hline
\end{longtable}

\subsubsection*{Entradas desde sistemas internos (cliente web, procesos internos y BD)}
\begin{longtable}{|p{5cm}|p{2.5cm}|p{2cm}|p{6cm}|}
\hline
	\textbf{Campo} & \textbf{Tipo de dato} & \textbf{Formato} & \textbf{Descripción} \\
\hline
\endfirsthead
\hline
	\textbf{Campo} & \textbf{Tipo de dato} & \textbf{Formato} & \textbf{Descripción} \\
\hline
\endhead
\hline
\endfoot
\hline
\endlastfoot

userId & Integer & id BD & Identificador del usuario (parámetros/relación). Por ejemplo: \texttt{GET }\icdendpoint{/api/flightPlans?userId=...} y \texttt{GET }\icdendpoint{/api/folders?userId=...}. \\
\hline
folderId & Integer/Null & id BD & Carpeta destino de un plan (opcional en creación/actualización). \\
\hline
customName & String & UTF-8 & Nombre del plan. En la UI suele derivar del nombre de fichero sin extensión. \\
\hline
fileContent & String & Texto & Contenido del plan cargado desde QGroundControl (enviado como texto). Se guarda en \texttt{flightPlan.fileContent}. \\
\hline
status & String & Enumerado & Estado de plan: \texttt{sin procesar}, \texttt{en cola}, \texttt{procesando}, \texttt{procesado}, \texttt{error}. \\
\hline
scheduledAt & String/Null & ISO-8601 & Fecha/hora programada del vuelo. Obligatoria para geoawareness y generación/envío de U-Plan. \\
\hline
uplan & Object/String & JSON & Detalles de U-Plan aportados por el usuario (opcional). Se utiliza para rellenar campos del U-Plan final. \\
\hline
authorizationStatus & String & Enumerado & Estado local de autorización (persistido en BD): \texttt{sin autorización} (por defecto), \texttt{aprobado}, \texttt{denegado}. \\
\hline
authorizationMessage & String/Object & Texto/JSON & Mensaje asociado a la autorización (errores, feedback, etc.). \\
\hline
externalResponseNumber & String/Null & Texto & Identificador devuelto por FAS tras envío del U-Plan; se usa para correlación del callback. \\
\hline
geoawarenessData & Object/Null & JSON & Campo JSON en BD para almacenar datos de geoawareness asociados al plan. \\
\hline
Auth: email/password & String & UTF-8 & Credenciales recibidas en \texttt{POST }\icdendpoint{/api/auth/signup} y \texttt{POST }\icdendpoint{/api/auth/login}. \\
\hline
Auth: Authorization header & String & Bearer JWT & Cabecera \texttt{Authorization: Bearer <token>} requerida en \texttt{GET }\icdendpoint{/api/user}. \\
\hline
\end{longtable}

\subsection*{Salidas}
Las salidas del sistema consisten en respuestas HTTP al cliente web y en peticiones HTTP a servicios externos (FAS y geoawareness).
\subsubsection*{Salidas hacia sistemas externos (FAS y Geoawareness)}
\begin{longtable}{|p{5cm}|p{2.5cm}|p{2cm}|p{6cm}|}
\hline
    \textbf{Campo} & \textbf{Tipo de dato} & \textbf{Formato} & \textbf{Descripción} \\
\hline
\endfirsthead
\hline
    \textbf{Campo} & \textbf{Tipo de dato} & \textbf{Formato} & \textbf{Descripción} \\
\hline
\endhead
\hline
\endfoot
\hline
\endlastfoot
HTTP request a FAS & JSON & application/json & \texttt{POST }\icdendpoint{<FAS_API_URL>} (si no se define \texttt{FAS\_ENDPOINT}, se usa el path \icdendpoint{/uplan}). Cuerpo: U-Plan. Cabecera: \texttt{Content-Type: application/json}. \\
\hline
SSE stream a cliente (proxy Geoawareness) & N/A & text/event-stream & El backend abre \texttt{ws://<GEOAWARENESS\_SERVICE\_IP>/ws/gas/\{USPACE\_ID\}} y retransmite cada mensaje al cliente via SSE (\texttt{GET /api/geoawareness/stream/\{uspaceId\}}). El navegador nunca abre WebSocket directo al servicio. \\
\hline
\end{longtable}

\subsubsection*{Salidas hacia sistemas internos (respuestas de la API al cliente web)}
\begin{longtable}{|p{5cm}|p{2.5cm}|p{2cm}|p{6cm}|}
\hline
    \textbf{Campo} & \textbf{Tipo de dato} & \textbf{Formato} & \textbf{Descripción} \\
\hline
\endfirsthead
\hline
    \textbf{Campo} & \textbf{Tipo de dato} & \textbf{Formato} & \textbf{Descripción} \\
\hline
\endhead
\hline
\endfoot
\hline
\endlastfoot

FlightPlan & JSON & -- & Objeto de plan devuelto por \icdendpoint{/api/flightPlans} (incluye campos de BD; en listados incluye \texttt{folder}). \\
\hline
Folder & JSON & -- & Objeto de carpeta devuelto por \icdendpoint{/api/folders}. \\
\hline
Machine & JSON & -- & Objeto de máquina devuelto por \icdendpoint{/api/machines}. \\
\hline
Bulk counters & Integer & -- & \texttt{createdCount} (creación bulk) y \texttt{count} (actualización bulk) en \icdendpoint{/api/flightPlans}. \\
\hline
CSV (individual) & String & CSV texto & \texttt{GET }\icdendpoint{/api/csvResult?id=...} devuelve \texttt{\{csvResult: "..."\}}. \\
\hline
CSV bulk items & Array & JSON & \texttt{POST }\icdendpoint{/api/csvResult} devuelve \texttt{\{items:[\{id,customName,csvResult\}]\}}. \\
\hline
Delete report & JSON & -- & \texttt{DELETE }\icdendpoint{/api/flightPlans} devuelve conteos: \texttt{deletedPlans}, \texttt{deletedCsvs}, \texttt{totalDeleted}. \\
\hline
U-Plan generado & JSON & application/json & \texttt{POST }\icdendpoint{/api/flightPlans/<id>/uplan} devuelve \texttt{\{uplan, authorizationMessage\}}. \\
\hline
Geoawareness result & JSON & -- & \texttt{POST }\icdendpoint{/api/flightPlans/<id>/geoawareness} devuelve \texttt{geozones}, \texttt{trajectory}, \texttt{uplan}, \texttt{hasConflicts}, \texttt{serviceAvailable}, \texttt{planName}. \\
\hline
Auth token & String & JWT & \texttt{POST }\icdendpoint{/api/auth/login} devuelve \texttt{\{token\}}. \\
\hline
\end{longtable}

\subsection*{Formato CSV de trayectoria (mínimo requerido)}
El conversor \texttt{trayToUplan} y el endpoint de geoawareness asumen un CSV con \textbf{encabezados} que incluya al menos las columnas: \texttt{SimTime}, \texttt{Lat}, \texttt{Lon}, \texttt{Alt}. Para geoawareness, la trayectoria para el mapa se extrae asumiendo que \texttt{Lat} y \texttt{Lon} están en las posiciones 2 y 3 (1-indexed) al separar por comas.

\subsection*{Formato U-Plan (salida generada, alto nivel)}
El U-Plan se genera a partir del CSV y \texttt{scheduledAt} y contiene una lista de \texttt{operationVolumes} (polígonos GeoJSON con ventana temporal y límites verticales). Los campos adicionales (operador, UAS, categorías, etc.) se rellenan desde \texttt{flightPlan.uplan} si se proporciona; en caso contrario, se generan valores por defecto/aleatorios.

\section{Protocolos y Transporte}
\begin{itemize}
    \item \textbf{Cliente web $\leftrightarrow$ UPPS}: HTTP (Next.js). Puerto expuesto típicamente: \texttt{3000}.
    \begin{itemize}
        \item Rutas API (\texttt{pages/api/*}): REST-like sobre HTTP con payload JSON.
        \item Límite de cuerpo (body parser): \texttt{50mb} en \icdendpoint{/api/flightPlans} y \texttt{10mb} en \icdendpoint{/api/csvResult}.
    \end{itemize}
    \item \textbf{UPPS $\leftrightarrow$ MySQL}: Prisma se conecta a MySQL mediante \texttt{DATABASE\_URL}. En el ejemplo de configuración, MySQL usa el puerto \texttt{3306}.
    \item \textbf{UPPS $\rightarrow$ FAS}: HTTP \texttt{POST} sin TLS a \texttt{http://<FAS\_IP>/<FAS\_ENDPOINT>}. Si \texttt{FAS\_ENDPOINT} no existe, se usa el path \texttt{uplan}. Content-Type: \texttt{application/json}.
    \item \textbf{FAS $\rightarrow$ UPPS}: HTTP \texttt{PUT} a \icdendpoint{/api/fas/<externalResponseNumber>} con JSON \texttt{\{state, message\}}.
    \item \textbf{Cliente web $\leftrightarrow$ Geoawareness (SSE proxy)}: La arquitectura utiliza SSE como capa de transporte entre el navegador y el backend, y WebSocket interno entre el backend y el servicio Geoawareness:
    \begin{itemize}
        \item \textbf{Navegador $\rightarrow$ Backend (SSE)}: \texttt{GET /api/geoawareness/stream/\{uspaceId\}} — EventSource HTTP estándar, compatible con HTTPS sin restricciones de mixed-content.
        \item \textbf{Backend $\rightarrow$ Geoawareness (WS interno)}: \texttt{ws://<GEOAWARENESS\_SERVICE\_IP>/ws/gas/\{uspaceId\}} — conexión WebSocket en la red interna del servidor; no requiere TLS.
        \item \textbf{Eventos SSE emitidos}: \texttt{connected} (WS abierto), \texttt{data} (mensaje del servicio), \texttt{error} (error WS), \texttt{close} (WS cerrado).
        \item \textbf{Reconexión automática}: el hook \texttt{useGeoawarenessWebSocket} implementa backoff exponencial (1s, 2s, 4s, 8s, 16s) sobre la conexión SSE.
        \item \textbf{Fallback}: Si la conexión SSE falla después de 5 intentos, el sistema carga datos estáticos de geozonas desde \texttt{/api/deprecated/geoawareness-geozones}.
    \end{itemize}
    \item \textbf{(Deprecated) UPPS $\rightarrow$ Geoawareness HTTP}: El endpoint HTTP \texttt{POST }\icdendpoint{/geozones\_searcher\_by\_volumes} está deprecado. Usar el proxy SSE para nuevas integraciones.
    \item \textbf{(Opcional) noVNC/QGroundControl}: la UI incluye un \texttt{iframe} a \texttt{http://localhost:6080/}. Este transporte depende de que noVNC esté levantado localmente.
    \item \textbf{Autenticación}:
    \begin{itemize}
        \item \texttt{POST }\icdendpoint{/api/auth/login} devuelve un JWT.
        \item \texttt{GET }\icdendpoint{/api/user} requiere \texttt{Authorization: Bearer <token>}.
        \item Los endpoints bajo \texttt{pages/api/*} no aplican verificación JWT en el estado actual del repositorio.
    \end{itemize}
\end{itemize}

\section{Identificadores y Convenciones}
\begin{itemize}
    \item \textbf{IDs de recursos (BD)}: todos los identificadores principales son enteros autoincrementales (\texttt{user.id}, \texttt{folder.id}, \texttt{flightPlan.id}, \texttt{machine.id}, \texttt{csvResult.id}).
    \item \textbf{Convención clave plan--CSV}: el borrado unificado asume \texttt{flightPlan.id == csvResult.id} (se eliminan CSVs con los mismos IDs que los planes borrados).
    \item \textbf{Correlación FAS}: \texttt{externalResponseNumber} (string) identifica el plan en el callback \icdendpoint{/api/fas/<externalResponseNumber>}.
    \item \textbf{Estados de procesamiento}:
    \begin{itemize}
        \item \texttt{flightPlan.status}: \texttt{sin procesar}, \texttt{en cola}, \texttt{procesando}, \texttt{procesado}, \texttt{error}.
        \item \texttt{machine.status}: \texttt{Disponible}, \texttt{Ocupada}.
        \item \texttt{flightPlan.authorizationStatus}: por defecto \texttt{sin autorización}; actualizaciones a \texttt{aprobado} o \texttt{denegado} tras callback FAS.
    \end{itemize}
    \item \textbf{Unidades/formatos}:
    \begin{itemize}
        \item \texttt{scheduledAt}: ISO-8601 (string) en API/BD.
        \item CSV: lat/lon en grados decimales; altitud numérica.
        \item U-Plan: geometrías GeoJSON con orden \texttt{[lon, lat]}.
        \item Altitudes en U-Plan: \texttt{uom="M"} y \texttt{reference="AGL"}.
    \end{itemize}
\end{itemize}

\section{Sincronización y Temporización}
\begin{itemize}
    \item \textbf{Asignación de planes a máquinas}: el proceso \texttt{traj-assigner.js} ejecuta asignación cada \textbf{10 ms} (\texttt{setInterval(assignNextPlan, 10)}).
    \item \textbf{Limpieza automática de CSV pequeños}: el mismo proceso ejecuta limpieza periódica cada \textbf{5 minutos} (\texttt{setInterval(cleanSmallCsvResults, 60 * 5000)}).
    \item \textbf{Regeneración automática de volúmenes}: el hook \texttt{useVolumeRegeneration} ejecuta verificación cada \textbf{30 segundos} para planes procesados sin volúmenes de operación.
    \item \textbf{Polling de estado de planes}: el hook \texttt{useFlightPlans} actualiza automáticamente el estado cada \textbf{5 segundos} cuando está habilitado.
    \item \textbf{Geoawareness WebSocket}:
    \begin{itemize}
        \item Reconexión con backoff exponencial: 1s, 2s, 4s, 8s, 16s (máximo 5 intentos).
        \item Fallback a HTTP después de agotar intentos.
    \end{itemize}
    \item \textbf{Normalización de tiempos de trayectoria}: 
    \begin{itemize}
        \item Parser CSV normaliza todos los tiempos para iniciar en \texttt{t=0s}.
        \item Se aplica en: \texttt{parseCSVToTrajectory} (GeoawarenessViewer, TrajectoryMapViewer).
        \item Se aplica en: \texttt{trayToUplan} para generación de volúmenes con tiempos correctos.
    \end{itemize}
    \item \textbf{Actualización de timestamps U-Plan}: \texttt{creationTime} y \texttt{updateTime} se actualizan al momento exacto del envío a FAS (ISO-8601 UTC).
    \item \textbf{JWT}: expiración del token de \textbf{1 día}.
    \item \textbf{Límites operacionales de la API (bulk)}:
    \begin{itemize}
        \item \icdendpoint{/api/csvResult}: máximo \texttt{5000} IDs por request (POST/DELETE bulk).
        \item \icdendpoint{/api/flightPlans}: el endpoint acepta hasta \texttt{100000} IDs en actualizaciones/borrados bulk; las actualizaciones por-item se trocean en chunks de 200 operaciones por transacción.
    \end{itemize}
\end{itemize}
\section{Gestión de Errores}
\begin{itemize}
    \item \textbf{Códigos HTTP usados}:
    \begin{itemize}
        \item \texttt{200}: operación correcta.
        \item \texttt{201}: recurso creado.
        \item \texttt{204}: borrado correcto sin contenido.
        \item \texttt{400}: parámetros inválidos o campos requeridos ausentes.
        \item \texttt{401}: autenticación requerida o token inválido (\icdendpoint{/api/user}).
        \item \texttt{404}: recurso no encontrado.
        \item \texttt{405}: método no permitido.
        \item \texttt{500}: error interno (BD/red/errores no controlados).
        \item \texttt{503} (v2.1): servicio temporalmente no disponible (FAS no responde después de 5 reintentos).
    \end{itemize}
    \item \textbf{Errores en integración FAS}:
    \begin{itemize}
        \item \textbf{Error 500 (Internal Server Error)}: Sistema reintenta automáticamente hasta 5 veces con backoff exponencial (1s, 2s, 4s, 8s, 16s).
        \item Si todos los reintentos fallan con 500: Plan permanece en estado \texttt{"sin autorización"} y la API devuelve \texttt{503 Service Unavailable}.
        \item \textbf{Errores 400/404/otros}: Marcan el plan como \texttt{authorizationStatus="denegado"} inmediatamente sin reintentos.
        \item Si el \texttt{POST} a FAS falla sin respuesta (timeout/red), marca \texttt{denegado} y devuelve \texttt{500} con mensaje.
        \item Callback FAS actualiza estado final: \texttt{aprobado} si \texttt{state=ACCEPTED}, \texttt{denegado} en caso contrario.
        \item Envío automático de email con resultado de autorización y UPLAN adjunto (v2.1).
    \end{itemize}
    \item \textbf{Geoawareness fallback}:
    \begin{itemize}
        \item Si el servicio no está configurado o no responde, el endpoint devuelve una respuesta mock y marca \texttt{serviceAvailable=false}.
    \end{itemize}
    \item \textbf{Recuperación del pipeline de trayectorias}:
    \begin{itemize}
        \item Un plan puede re-ponerse en \texttt{en cola} desde la UI (\texttt{PUT }\icdendpoint{/api/flightPlans}).
        \item \texttt{traj-assigner.js} reencola automáticamente planes procesados cuyo CSV asociado es menor a 2 KB (criterio implementado en BD).
        \item \texttt{machine-cleaner.js} borra todas las máquinas y reencola planes en \texttt{procesando} a \texttt{en cola}.
    \end{itemize}
    \item \textbf{Validación de emails (v2.1)}:
    \begin{itemize}
        \item Código de verificación: 6 dígitos, válido por 24 horas.
        \item Token de reset: válido por 1 hora desde generación.
        \item Errores de formato detectados con Zod schemas antes de persistencia.
    \end{itemize}
\end{itemize}

\section{Versionado y Control de Cambios}
\begin{itemize}
    \item \textbf{Versión del ICD}: \texttt{V\_2.2.0}.
    \item \textbf{Versión de la aplicación}: \texttt{v2.2.0 SPIRIT} (Sprints 1--3: 26 funcionalidades nuevas, proxy SSE de Geoawareness, visores 3D completos, gestión de usuario, sistema de contacto, restricciones EU).
    \item \textbf{Cambios en v2.2.0 SPIRIT}:
    \begin{itemize}
        \item \textbf{Auth}: Eliminación de validación DNS en registro; caducidad de código de verificación a 3 min; doble-verificación safeguard.
        \item \textbf{UI/UX}: Tema del sistema por defecto; toasts opacos; validación de archivo U-Plan en carga; tooltips \texttt{FieldHelp} en formularios.
        \item \textbf{Perfil de usuario}: Campos \texttt{firstName}, \texttt{lastName}, \texttt{phone}; endpoints \texttt{GET/PATCH /api/user/profile}; pre-relleno automático en U-Plan form.
        \item \textbf{Email}: Notificación al cancelar plan aprobado; email de confirmación en sistema de tickets.
        \item \textbf{Plan Definition (ex Plan Generator)}: Botón OK y Enter en edición de waypoints; botón ``Añadir Waypoint'' en sidebar; fix zoom reset SCAN; generación automática de volúmenes antes de descarga; navegación directa a Plan Authorization tras subir plan.
        \item \textbf{FAS}: Coloreado selectivo de volúmenes conflictivos (intersección 2D); razón FAS parseada en lenguaje natural; sincronización del campo \texttt{state} del U-Plan con la decisión FAS.
        \item \textbf{Restricciones regulatorias EU}: Tabla C0--C6 con MTOM y dimensión máxima; validación en formulario.
        \item \textbf{Visores 3D CesiumJS}: \texttt{Cesium3DModal} (volúmenes operativos + OSM Buildings + slider 4D); \texttt{Denial3DModal} (primas 3D de geozonas conflictivas); \texttt{AuthorizationResultModal} (modal unificado 2D/3D); \texttt{Trajectory3DViewer} (dron animado + RAF + AGL unificado); InfoBox dark-theme fix en todos los visores.
        \item \textbf{Contacto}: Página \texttt{/contact-us}; endpoint \texttt{POST /api/contact}; tickets \texttt{UPPS-AAAAMMDD-XXXX}.
        \item \textbf{Renombrado de páginas}: Plan Generator $\rightarrow$ \texttt{/plan-definition}, Trajectory Generator $\rightarrow$ \texttt{/plan-authorization}.
        \item \textbf{Proxy SSE Geoawareness}: Backend conecta \texttt{ws://} internamente y sirve SSE al navegador via \texttt{GET /api/geoawareness/stream/[uspaceId]}; elimina mixed-content en HTTPS; depreca \texttt{NEXT\_PUBLIC\_GEOAWARENESS\_SERVICE\_IP}.
        \item \textbf{Fix Cesium assets}: \texttt{postinstall}/\texttt{prebuild} copian directorio completo \texttt{Build/Cesium/} (incluyendo \texttt{Cesium.js}) a \texttt{public/cesium/}.
        \item \textbf{Nuevos endpoints}: \texttt{GET/PATCH /api/user/profile}, \texttt{POST /api/contact}, \texttt{GET /api/cesium/token}, \texttt{GET /api/geoawareness/stream/[uspaceId]}.
    \end{itemize}
    \item \textbf{Cambios en v2.1.0 SPIRIT}:
    \begin{itemize}
        \item \textbf{Sistema de emails completo}: Integración con MailerSend para verificación de email, reset de contraseña y notificaciones de autorización FAS.
        \item \textbf{Importación de UPLANs externos}: Soporte completo para drag-and-drop y upload de archivos .json con UPLANs pregenerados.
        \item \textbf{Selector de U-Space interactivo}: Modal con mapa para asignar U-Space a UPLANs externos antes de verificación de geoawareness.
        \item \textbf{Visualización de denegaciones}: Modal de mapa especializado para mostrar volúmenes conflictivos y razones de denegación.
        \item \textbf{Edición de waypoints SCAN}: Popups editables en mapas con campos para coordenadas (lat/lng).
        \item \textbf{Mejoras de tema UI}: Tema oscuro completo en popups Leaflet y componente ScanPatternGenerator.
        \item \textbf{Descarga de U-Plan}: Botón para descargar JSON del U-Plan procesado.
        \item \textbf{Reintentos FAS 500}: Lógica de reintentos con backoff exponencial para errores internos del FAS.
        \item \textbf{Correcciones de formato}: Conversión automática de scheduledAt a ISO-8601 con timezone para UPLANs externos.
        \item \textbf{Nuevos endpoints}:
        \begin{itemize}
            \item \texttt{POST /api/auth/verify-email} -- Verificación de email con código
            \item \texttt{POST /api/auth/request-password-reset} -- Solicitud de reset de contraseña
            \item \texttt{POST /api/auth/reset-password} -- Reset de contraseña con token
        \end{itemize}
    \end{itemize}
    \item \textbf{Cambios en v2.0.0 SPIRIT}:
    \begin{itemize}
        \item \textbf{UI/UX Mejorado}: Sistema de temas dinámico (claro/oscuro) con CSS variables y persistencia.
        \item \textbf{Normalización de tiempos}: Trayectorias CSV normalizadas para iniciar en t=0s, eliminando offsets temporales.
        \item \textbf{Auto-regeneración de volúmenes}: Hook \texttt{useVolumeRegeneration} activo por defecto, verifica cada 30s planes procesados sin volúmenes.
        \item \textbf{Viewer de trayectorias mejorado}: Componente \texttt{TrajectoryMapViewer} con mapas Leaflet interactivos y análisis detallado.
        \item \textbf{Geoawareness avanzado}: Auto-centrado de mapa en trayectoria, marcadores reducidos (5/3px), overlays oscuros consistentes.
        \item \textbf{WebSocket optimizado}: Manejo de errores mejorado, no muestra errores durante conexión inicial, fallback híbrido.
        \item \textbf{Gestión de folders}: Drag-and-drop completo, sección de planes huérfanos, operaciones de renombrado y movimiento.
        \item \textbf{Timestamps dinámicos}: \texttt{creationTime} y \texttt{updateTime} del U-Plan se actualizan al momento del envío a FAS.
        \item \textbf{Workflow de 5 pasos}: Select → DateTime → Process → Geoawareness → Authorize con indicadores visuales.
        \item \textbf{Arquitectura modular}: Componentes organizados en \texttt{app/components/flight-plans/} con hooks especializados.
        \item \textbf{Endpoints optimizados}:
        \begin{itemize}
            \item \texttt{POST /api/flightPlans/regenerate-volumes}: Regeneración automática de volúmenes.
            \item \texttt{POST /api/flightPlans/[id]/reset}: Reset completo de plan a estado inicial.
            \item Mejoras en filtrado de planes procesados (\texttt{status='procesado'}, \texttt{csvResult=1}).
        \end{itemize}
    \end{itemize}
    \item \textbf{Cambios en v1.0.0} (referencia histórica):
    \begin{itemize}
        \item Migración de Geoawareness de HTTP polling a WebSocket para obtención de geozonas en tiempo real.
        \item Nuevo endpoint WebSocket: \texttt{ws://<IP>/ws/gas/\{USPACE\_ID\}}.
        \item Nuevo formato de mensaje de geoawareness con 3 bloques estructurados.
        \item Deprecación del endpoint HTTP \texttt{POST /geozones\_searcher\_by\_volumes}.
        \item Sistema de fallback híbrido (WebSocket primario, HTTP legacy como backup).
    \end{itemize}
    \item \textbf{Control de cambios de interfaz}:
    \begin{itemize}
        \item El sistema introduce un patrón de \textit{API unificada} donde \icdendpoint{/api/flightPlans} y \icdendpoint{/api/csvResult} concentran operaciones individuales y bulk.
        \item El endpoint \icdendpoint{/api/flightPlans/[id]} se declara \textit{deprecated} (placeholder) y no debe usarse para nuevas integraciones.
        \item Para compatibilidad hacia atrás, se recomienda mantener campos opcionales (p.ej. \texttt{geoawarenessData}, \texttt{authorizationMessage}) sin romper respuestas existentes.
        \item El endpoint HTTP de geoawareness está movido a \icdendpoint{/api/deprecated/geoawareness-geozones} y solo debe usarse como fallback.
        \item Nuevo sistema de regeneración automática opera de forma transparente sin intervención del usuario.
    \end{itemize}
\end{itemize}

\section{Integración con el Servicio Geoawareness}
\label{sec:geoawareness-ws-format}

\subsection{Arquitectura del Proxy SSE}
A partir de la versión 2.2.0, la comunicación con el servicio Geoawareness ya no se realiza con una conexión WebSocket directa desde el navegador, sino a través de un proxy SSE alojado en el propio backend UPPS:

\begin{itemize}
    \item \textbf{Motivación}: Una conexión \texttt{ws://} iniciada desde una página servida por HTTPS es bloqueada por los navegadores (política de mixed-content). El proxy SSE elimina esta restricción sin requerir TLS en el servicio Geoawareness.
    \item \textbf{Flujo}:
    \begin{enumerate}
        \item El navegador abre \texttt{GET /api/geoawareness/stream/\{uspaceId\}} como \texttt{EventSource} (SSE sobre HTTPS).
        \item El backend Next.js (runtime Node.js) abre \texttt{ws://<GEOAWARENESS\_SERVICE\_IP>/ws/gas/\{uspaceId\}} de forma interna usando la librería \texttt{ws}.
        \item Cada mensaje WebSocket recibido del servicio se retransmite al navegador como un evento SSE (tipo \texttt{data}).
        \item Si el navegador cierra la conexión SSE, el backend cierra el WebSocket interno para evitar fugas de recursos.
    \end{enumerate}
    \item \textbf{Evento SSE \texttt{connected}}: emitido cuando el WS interno se abre correctamente.
    \item \textbf{Evento SSE \texttt{error}}: emitido si el WS interno reporta un error; el stream SSE se cierra.
    \item \textbf{Evento SSE \texttt{close}}: emitido cuando el WS interno cierra; el stream SSE se cierra.
    \item \textbf{Reconexión}: gestionada por el hook \texttt{useGeoawarenessWebSocket} con backoff exponencial sobre la conexión SSE.
\end{itemize}

\subsection{Endpoint proxy SSE}
\begin{itemize}
    \item \textbf{URL}: \texttt{GET /api/geoawareness/stream/\{uspaceId\}}
    \item \textbf{Parámetro path}: \texttt{uspaceId} -- Identificador del U-space (ej: \texttt{VLCUspace})
    \item \textbf{Content-Type respuesta}: \texttt{text/event-stream}
    \item \textbf{Headers}: \texttt{Cache-Control: no-cache, no-transform}, \texttt{X-Accel-Buffering: no}
    \item \textbf{Autenticación}: ninguna (la autenticación se gestiona en el frontend; el endpoint es interno al mismo origen)
\end{itemize}

\subsection{States de Conexión (hook)}
El hook \texttt{useGeoawarenessWebSocket} gestiona los siguientes estados:
\begin{itemize}
    \item \texttt{connecting}: Estableciendo conexión SSE al backend
    \item \texttt{connected}: El backend confirmó apertura del WS, recibiendo datos
    \item \texttt{disconnected}: Conexión SSE cerrada
    \item \texttt{error}: Error en la conexión SSE o WS interno
\end{itemize}

\subsection{Reconexión Automática}
El sistema implementa reconexión automática con backoff exponencial sobre la conexión SSE:
\begin{itemize}
    \item Intentos máximos: 5
    \item Delays: 1s $\rightarrow$ 2s $\rightarrow$ 4s $\rightarrow$ 8s $\rightarrow$ 16s
    \item Tras agotar intentos: Fallback a endpoint HTTP deprecado
\end{itemize}

\subsection{Formato de Mensaje (inalterado)}
El payload JSON de cada evento SSE es idéntico al que enviaba el antiguo WebSocket directo. Se estructura en 3 bloques principales:

\subsubsection{Bloque 1: Campos de Control}
\begin{longtable}{|p{4cm}|p{2.5cm}|p{8.5cm}|}
\hline
\textbf{Campo} & \textbf{Tipo} & \textbf{Descripción} \\
\hline
\endfirsthead
\hline
\textbf{Campo} & \textbf{Tipo} & \textbf{Descripción} \\
\hline
\endhead
\hline
\endfoot
\hline
\endlastfoot
uspace\_identifier & String & Identificador único del U-space (ej: ``VLCUspace'') \\
\hline
timestamp & String (ISO-8601) & Momento de generación del mensaje \\
\hline
\end{longtable}

\subsubsection{Bloque 2: Datos del U-space (uspace\_data)}
Objeto GeoJSON Feature que describe el área del U-space:

\begin{longtable}{|p{4cm}|p{2.5cm}|p{8.5cm}|}
\hline
\textbf{Campo} & \textbf{Tipo} & \textbf{Descripción} \\
\hline
\endfirsthead
\hline
\textbf{Campo} & \textbf{Tipo} & \textbf{Descripción} \\
\hline
\endhead
\hline
\endfoot
\hline
\endlastfoot
type & String & Siempre ``Feature'' \\
\hline
id & Integer & Identificador numérico del U-space \\
\hline
bbox & Array[Float] & Bounding box [minLon, minLat, maxLon, maxLat] \\
\hline
name & String & Nombre legible del U-space \\
\hline
source & String & Origen de los datos \\
\hline
geometry.type & String & Tipo de geometría (``Polygon'', ``MultiPolygon'') \\
\hline
geometry.coordinates & Array & Coordenadas GeoJSON [[[lon,lat],...]] \\
\hline
geometry.verticalReference & Object & Límites verticales (ver tabla siguiente) \\
\hline
geometry.sub\_type & String & Subtipo de geometría \\
\hline
geometry.radius & Float & Radio para geometrías circulares \\
\hline
properties.identifier & String & Identificador formal del U-space \\
\hline
properties.country & String & País (código ISO) \\
\hline
properties.type & String & Tipo de zona \\
\hline
properties.region & String & Región geográfica \\
\hline
\end{longtable}

\subsubsection{Referencia Vertical (verticalReference)}
\begin{longtable}{|p{4cm}|p{2.5cm}|p{8.5cm}|}
\hline
\textbf{Campo} & \textbf{Tipo} & \textbf{Descripción} \\
\hline
\endfirsthead
\hline
\textbf{Campo} & \textbf{Tipo} & \textbf{Descripción} \\
\hline
\endhead
\hline
\endfoot
\hline
\endlastfoot
upper & Float & Límite superior de altitud \\
\hline
upperReference & String & Referencia del límite superior (``AGL'', ``AMSL'') \\
\hline
lower & Float & Límite inferior de altitud \\
\hline
lowerReference & String & Referencia del límite inferior (``AGL'', ``AMSL'') \\
\hline
uom & String & Unidad de medida (``M'' para metros) \\
\hline
\end{longtable}

\subsubsection{Condiciones de Restricción (restrictionConditions)}
\begin{longtable}{|p{4cm}|p{2.5cm}|p{8.5cm}|}
\hline
\textbf{Campo} & \textbf{Tipo} & \textbf{Descripción} \\
\hline
\endfirsthead
\hline
\textbf{Campo} & \textbf{Tipo} & \textbf{Descripción} \\
\hline
\endhead
\hline
\endfoot
\hline
\endlastfoot
uasClass & Array[String] & Clases de UAS permitidas (``C0'', ``C1'', ``C2'', etc.) \\
\hline
authorized & String & Tipo de autorización requerida \\
\hline
uasCategory & Array[String] & Categorías de operación permitidas \\
\hline
uasOperationMode & Array[String] & Modos de operación (``VLOS'', ``BVLOS'') \\
\hline
maxNoise & Float & Nivel máximo de ruido permitido (dB) \\
\hline
specialOperation & String & Operaciones especiales permitidas \\
\hline
photograph & String & Restricciones de fotografía \\
\hline
\end{longtable}

\subsubsection{Autoridad de Zona (zoneAuthority)}
\begin{longtable}{|p{4cm}|p{2.5cm}|p{8.5cm}|}
\hline
\textbf{Campo} & \textbf{Tipo} & \textbf{Descripción} \\
\hline
\endfirsthead
\hline
\textbf{Campo} & \textbf{Tipo} & \textbf{Descripción} \\
\hline
\endhead
\hline
\endfoot
\hline
\endlastfoot
name & String & Nombre de la autoridad \\
\hline
service & String & Tipo de servicio \\
\hline
SiteURL & String & URL del sitio web \\
\hline
email & String & Email de contacto \\
\hline
phone & String & Teléfono de contacto \\
\hline
purpose & String & Propósito de la zona \\
\hline
intervalBefore & String & Tiempo de antelación requerido \\
\hline
contactName & String & Nombre del contacto \\
\hline
\end{longtable}

\subsubsection{Aplicabilidad Limitada (limitedApplicability)}
\begin{longtable}{|p{4cm}|p{2.5cm}|p{8.5cm}|}
\hline
\textbf{Campo} & \textbf{Tipo} & \textbf{Descripción} \\
\hline
\endfirsthead
\hline
\textbf{Campo} & \textbf{Tipo} & \textbf{Descripción} \\
\hline
\endhead
\hline
\endfoot
\hline
\endlastfoot
startDatetime & String (ISO-8601) & Inicio de validez de la restricción \\
\hline
endDatetime & String (ISO-8601) & Fin de validez de la restricción \\
\hline
schedule.day & Array[String] & Días de la semana aplicables \\
\hline
schedule.startTime & String & Hora de inicio (HH:MM) \\
\hline
schedule.startEvent & String & Evento de inicio \\
\hline
schedule.endTime & String & Hora de fin (HH:MM) \\
\hline
schedule.endEvent & String & Evento de fin \\
\hline
\end{longtable}

\subsubsection{Bloque 3: Geozonas (geozones)}
FeatureCollection GeoJSON con las geozonas asociadas al U-space. Cada Feature tiene la misma estructura que \texttt{uspace\_data} (geometría, properties con restrictionConditions, zoneAuthority, limitedApplicability).

\subsection{Ejemplo de Mensaje (payload SSE)}
El campo \texttt{data} de cada evento SSE contiene el mismo JSON que enviaba el antiguo WebSocket:
\begin{lstlisting}[style=json]
{
  "uspace_identifier": "VLCUspace",
  "timestamp": "2026-02-03T10:00:00Z",
  "uspace_data": {
    "type": "Feature",
    "id": 1,
    "name": "Valencia U-space",
    "geometry": {
      "type": "Polygon",
      "coordinates": [[[-0.5, 39.4], [-0.3, 39.4], [-0.3, 39.6], [-0.5, 39.6], [-0.5, 39.4]]],
      "verticalReference": {
        "upper": 120, "upperReference": "AGL",
        "lower": 0, "lowerReference": "AGL", "uom": "M"
      }
    },
    "properties": {
      "identifier": "USP-ESP-VLC-01",
      "country": "ESP",
      "type": "uspace"
    }
  },
  "geozones": {
    "type": "FeatureCollection",
    "features": [
      {
        "type": "Feature",
        "geometry": { "type": "Polygon", "coordinates": [...] },
        "properties": {
          "identifier": "GZ_001",
          "type": "prohibited",
          "restrictionConditions": { "uasClass": ["C0", "C1"], ... },
          "zoneAuthority": { "name": "AESA", "email": "...", ... },
          "limitedApplicability": { "startDatetime": "...", ... }
        }
      }
    ]
  }
}
\end{lstlisting}

\subsection{Sistema de Fallback}
Si la conexión SSE falla después de 5 intentos de reconexión, el sistema activa automáticamente un fallback HTTP:

\begin{itemize}
    \item \textbf{Endpoint fallback}: \texttt{GET /api/deprecated/geoawareness-geozones}
    \item \textbf{Fuente de datos}: \texttt{lib/geozones/geozones\_dataFrame.geojson}
    \item \textbf{Formato}: GeoJSON FeatureCollection (formato legacy)
    \item \textbf{Normalización}: El componente \texttt{geozone-normalizer.ts} convierte ambos formatos a una estructura común
\end{itemize}

\section*{Anexos}
\subsection*{A. Variables de entorno (según \texttt{.env.example} y código)}
\begin{itemize}
    \item \texttt{DATABASE\_URL}: conexión MySQL (ejemplo: \texttt{mysql://USER@HOST:3306/upps}).
    \item \texttt{FAS\_API\_URL}: URL completa del servicio FAS (incluye protocolo, host, puerto y path).
    \item \texttt{GEOAWARENESS\_SERVICE\_IP}: host:puerto del servicio geoawareness (servidor). Usado por el proxy SSE para abrir la conexión WebSocket interna.
    \item \texttt{GEOAWARENESS\_ENDPOINT} (opcional): prefijo del path WebSocket (por defecto: \texttt{ws/gas/}).
    \item \texttt{NEXT\_PUBLIC\_GEOAWARENESS\_SERVICE\_IP}: \textbf{deprecado en v2.2.0}. Ya no es necesario; el navegador no abre WebSocket directamente.
    \item \texttt{MAILERSEND\_API\_KEY}: API key de MailerSend para envío de emails (verificación, reset, notificaciones).
    \item \texttt{MAILERSEND\_SENDER\_EMAIL} (opcional): Email del remitente (por defecto: \texttt{UPPS@sna-upv.com}).
    \item \texttt{MAILERSEND\_SENDER\_NAME} (opcional): Nombre del remitente (por defecto: \texttt{UPPS}).
    \item \texttt{NEXT\_PUBLIC\_APP\_URL}: URL base de la aplicación para links en emails (por defecto: \texttt{http://localhost:3001}).
    \item \texttt{JWT\_SECRET} (opcional, recomendado): secreto para JWT (si no existe, el sistema usa un valor por defecto en código).
    \item \texttt{NEXT\_PUBLIC\_GENERATE\_RANDOM\_UPLAN\_DATA} (opcional): Si es 'true', genera datos aleatorios para campos faltantes en U-Plan (solo desarrollo).
\end{itemize}

\subsection*{B. Endpoints HTTP implementados (resumen)}
\begin{itemize}
    \item \texttt{GET }\icdendpoint{/api/flightPlans?userId=<id>}
    \item \texttt{POST }\icdendpoint{/api/flightPlans} (individual o bulk \texttt{items}; soporta importación UPLAN externo con \texttt{type="external\_uplan"})
    \item \texttt{PUT }\icdendpoint{/api/flightPlans} (\texttt{\{id,data\}} o bulk \texttt{\{ids,data\}} / \texttt{\{items\}})
    \item \texttt{DELETE }\icdendpoint{/api/flightPlans} (\texttt{\{id\}} o bulk \texttt{\{ids\}}; elimina también CSV asociado; envía cancelación a FAS si aprobado)
    \item \texttt{POST }\icdendpoint{/api/flightPlans/regenerate-volumes} (v2.0: regeneración automática de volúmenes)
    \item \texttt{POST }\icdendpoint{/api/flightPlans/<id>/reset} (v2.0: reset completo de plan a estado inicial)
    \item \texttt{POST }\icdendpoint{/api/flightPlans/<id>/generate-volumes} (generación bajo demanda de volúmenes desde CSV)
    \item \texttt{GET }\icdendpoint{/api/csvResult?id=<id>}
    \item \texttt{POST }\icdendpoint{/api/csvResult} (bulk \texttt{\{ids\}})
    \item \texttt{DELETE }\icdendpoint{/api/csvResult} (\texttt{\{id\}} o bulk \texttt{\{ids\}})
    \item \texttt{POST }\icdendpoint{/api/flightPlans/<id>/geoawareness} (verifica plan contra geozonas)
    \item \texttt{POST }\icdendpoint{/api/flightPlans/<id>/uplan} (genera y envía U-Plan a FAS; detección automática de UPLAN externo; reintentos en error 500)
    \item \texttt{PUT }\icdendpoint{/api/fas/<externalResponseNumber>} (callback de FAS; envía email de notificación con UPLAN adjunto)
    \item \texttt{DELETE }\icdendpoint{/api/uplan_cancelation/<externalResponseNumber>} (cancelación de UPLAN aprobado en FAS)
    \item \texttt{GET/POST }\icdendpoint{/api/folders?userId=<id>}
    \item \texttt{GET/PUT/DELETE }\icdendpoint{/api/folders/<id>}
    \item \texttt{POST }\icdendpoint{/api/machines} y \texttt{GET }\icdendpoint{/api/machines?name=<name>}
    \item \texttt{PUT }\icdendpoint{/api/machines/<id>}
    \item \texttt{POST }\icdendpoint{/api/auth/signup} (registro con envío de email de verificación)
    \item \texttt{POST }\icdendpoint{/api/auth/login}
    \item \texttt{GET }\icdendpoint{/api/user}
    \item \texttt{POST }\icdendpoint{/api/auth/verify-email} (v2.1: verificación de email con código de 6 dígitos o token)
    \item \texttt{POST }\icdendpoint{/api/auth/request-password-reset} (v2.1: solicitud de reset con envío de email)
    \item \texttt{POST }\icdendpoint{/api/auth/reset-password} (v2.1: reset de contraseña con token)
    \item \texttt{GET }\icdendpoint{/api/geoawareness/uspaces} (proxy para lista de U-spaces)
    \item \texttt{GET }\icdendpoint{/api/geoawareness/stream/[uspaceId]} (v2.2.0: proxy SSE de Geoawareness; retransmite WebSocket interno como SSE)
    \item \texttt{GET }\icdendpoint{/api/deprecated/geoawareness-geozones} (fallback HTTP, deprecated)
\end{itemize}

\subsection*{C. Endpoints de streaming}
\begin{itemize}
    \item \texttt{GET /api/geoawareness/stream/\{uspaceId\}} -- Proxy SSE de Geoawareness (v2.2.0). El backend abre internamente \texttt{ws://<GEOAWARENESS\_SERVICE\_IP>/ws/gas/\{uspaceId\}} y retransmite los mensajes al navegador como eventos SSE.
    \item \textbf{(Servidor interno, no expuesto)}: \texttt{ws://<GEOAWARENESS\_SERVICE\_IP>/ws/gas/\{USPACE\_ID\}} -- Solo accedido por el backend.
\end{itemize}

\subsection*{D. Flujo de alto nivel (texto)}
\begin{enumerate}
    \item El usuario abre \textbf{Plan Definition} (\texttt{/plan-definition}) y selecciona un U-space.
    \item El cliente web conecta al proxy SSE backend \texttt{/api/geoawareness/stream/\{USPACE\_ID\}} que a su vez abre \texttt{ws://.../ws/gas/\{USPACE\_ID\}} para recibir geozonas en tiempo real.
    \item El usuario define waypoints sobre el mapa, visualizando las geozonas como referencia.
    \item El usuario sube planes: \texttt{POST }\icdendpoint{/api/flightPlans} con \texttt{fileContent} y \texttt{status="sin procesar"}.
    \item Tras subir, la app navega automáticamente a \textbf{Plan Authorization} (\texttt{/plan-authorization?selectPlan=<id>}) y auto-selecciona el plan importado.
    \item \textbf{Workflow de 5 pasos} (Plan Authorization):
    \begin{enumerate}
        \item \textbf{Select}: Usuario selecciona un plan de vuelo de la lista.
        \item \textbf{DateTime}: Usuario establece fecha/hora programada (\texttt{scheduledAt}).
        \item \textbf{Process}: Usuario procesa el plan (marca como \texttt{"en cola"} mediante \texttt{PUT }\icdendpoint{/api/flightPlans}).
        \item \textbf{Geoawareness}:
        \begin{itemize}
            \item Sistema regenera volúmenes automáticamente si faltan (cada 30s).
            \item Usuario revisa U-Plan generado y verifica contra geozonas via WebSocket.
            \item Usuario visualiza trayectoria procesada en mapa interactivo.
        \end{itemize}
        \item \textbf{Authorize}: Usuario envía U-Plan a FAS para autorización.
    \end{enumerate}
    \item \texttt{traj-assigner.js} asigna planes \texttt{"en cola"} a máquinas \texttt{"Disponible"} y marca el plan \texttt{"procesando"}.
    \item Los workers externos generan el CSV con normalización de tiempos y actualizan BD (\texttt{csvResult} y estado del plan).
    \item Check Geoawareness: El usuario visualiza la trayectoria contra geozonas actualizadas via WebSocket en tiempo real.
    \item Envío a autorización: \texttt{POST }\icdendpoint{/api/flightPlans/<id>/uplan} (con timestamps actualizados) y callback FAS a \texttt{PUT }\icdendpoint{/api/fas/<externalResponseNumber>}.
    \item Sistema regenera volúmenes automáticamente si el plan se resetea o pierde sus volúmenes.
\end{enumerate}

\subsection*{E. Arquitectura de Componentes (v2.0)}
La aplicación se estructura en componentes modulares organizados por funcionalidad:

\subsubsection*{Componentes principales}
\begin{itemize}
    \item \texttt{FlightPlansUploader}: Interfaz principal del \textbf{Plan Authorization} con workflow guiado.
    \item \texttt{PlanGenerator} / \textbf{Plan Definition}: Generador de planes de vuelo con mapa interactivo y waypoints (página \texttt{/plan-definition}).
    \item \texttt{ProcessingWorkflow}: Indicador visual del workflow de 5 pasos.
    \item \texttt{FolderList}: Gestión de carpetas con drag-and-drop.
    \item \texttt{FlightPlanCard}: Tarjeta de plan con operaciones individuales.
    \item \texttt{DateTimePicker}: Selector de fecha/hora con soporte UTC.
\end{itemize}

\subsubsection*{Componentes de visualización}
\begin{itemize}
    \item \texttt{Trajectory3DViewer}: Visor 3D de trayectoria con dron animado (Cesium), altitudes AGL.
    \item \texttt{GeoawarenessViewer}: Visualizador de trayectoria sobre geozonas con SSE en tiempo real.
    \item \texttt{Cesium3DModal}: Visor 3D de volúmenes de operación U-Plan con slider temporal 4D.
    \item \texttt{AuthorizationResultModal}: Modal unificado de resultado de autorización (2D + 3D).
    \item \texttt{Denial3DModal}: Visor 3D de denegación con geozonas conflictivas.
    \item \texttt{WaypointMapModal}: Modal para preview de waypoints en plan.
    \item \texttt{UplanViewModal}: Visualizador de volúmenes de operación en mapa.
    \item \texttt{UplanFormModal}: Formulario de edición de campos U-Plan (con tooltips \texttt{FieldHelp}).
    \item \texttt{GeozoneLayer}: Capa de geozonas con colores por tipo y opacidad configurable.
\end{itemize}

\subsubsection*{Hooks especializados}
\begin{itemize}
    \item \texttt{useFlightPlans}: Gestión de planes con polling opcional (5s).
    \item \texttt{useFolders}: Gestión de carpetas.
    \item \texttt{useGeoawarenessWebSocket}: Conexión SSE con reconexión automática (v2.2.0: proxy backend).
    \item \texttt{useVolumeRegeneration}: Regeneración automática de volúmenes (30s).
    \item \texttt{usePolling}: Polling genérico con control de errores.
    \item \texttt{useTheme}: Gestión del tema claro/oscuro con persistencia.
    \item \texttt{useToast}: Notificaciones toast con estados y retry.
\end{itemize}

\subsubsection*{Componentes UI reutilizables (v2.2.0)}
\begin{itemize}
    \item \texttt{FieldHelp}: Icono \texttt{?} con tooltip al click; explica acrónimos de aviación en formularios U-Plan.
\end{itemize}

\subsubsection*{Sistema de temas (v2.0)}
\begin{itemize}
    \item CSS variables dinámicas: \texttt{--bg-primary}, \texttt{--surface-primary}, \texttt{--text-primary}, etc.
    \item Cambio instantáneo sin recarga de página.
    \item Persistencia en \texttt{localStorage}.
    \item Soporte para \texttt{MutationObserver} en componentes dinámicos (header, footer).
    \item Variables de estado: \texttt{--status-success-bg}, \texttt{--status-error-bg}, etc.
\end{itemize}

\end{document}

